\section{Question 1}

\begin{question}
    A commutative ring $R$ with identity is called a local ring if it has a unique maximal ideal.
\end{question}

\subsection{Part a}

\begin{question}
    Prove that if $R$ is a local ring with maximal ideal $M$, then every element of $R-M$ is a unit.
\end{question}

\begin{answer}
    \begin{proof}
        Let $x \in R-M$ and assume $x$ is not a unit, then $(x) \neq (1) = R$. By the Theorem proved in class using Zorn's Lemma, we know that since $(x)$ is an ideal, then it is contained in some maximal ideal of $R$. Because $R$ is a local ring (i.e. $M$ is the only maximal ring of $R$), then $(x) \subseteq M$, which means $x \in M$. This is a contradiction. Thus, every element in $R-M$ must be a unit.
    \end{proof}
\end{answer}

\subsection{Part b}

\begin{question}
    Conversely, if the set of non-units in $R$ forms an ideal $M$, prove that $R$ is a local ring and that $M$ is its maximal ideal.
\end{question}

\begin{answer}
    \begin{proof}
        First, I want to show $M$ is maximal. Assume for contradiction that there exist an ideal $M' \neq R$, and $M \subset M'$. Then, since $M$ contains all non-units, then there must be a unit $x$ in $M$. Then, $x^{-1} \in M$, and therefore $(1) \in M$. Hence, $M = R$, which is a contradiction. Therefore, $M$ should be maximal.
        
        Next, I want to show that $M$ is the unique maximal ideal in $R$. Again, suppose for contradiction, there exist another maximal ideal $M^* \neq M$, then $M^*$ must contain at least one element that is  not in $M$. By construction, there must be one unit in $M^*$, then for the same reason as the previous proof, we know that $M^* = R$. This contradicts the assumption that $M$ is a maximal ideal. Hence, $M$ would be the only maximal ideal, and therefore $R$ is a local ring.
    \end{proof}
\end{answer}

\subsection{Part c}

\begin{question}
    Prove that the ring of all rational numbers whose denominator is odd is a local ring whose unique maximal ideal is the principal ideal generated by 2 .
\end{question}

\begin{answer}
    \begin{proof}
        Let $R = \{\tfrac{a}{b} \in \mathbb{Q} \mid \gcd(a,b) = 1, a,b \in \mathbb{Z}, \text{ and } 2 \nmid b\}$. Since the inverse of an element $x$ in $R$ must follow $x \cdot x^{-1} = 1$, the inverse should follow the reciprocal of the element. However, only element in $R$ but not in $(2)$ has well-defined inverse, because if $x \in (2)$, then $x^{-1} = \tfrac{1}{x}$ will be a element with an even denominator. Thus, we know that $(2)$ is the set of non-units in $R$, and for all $y \in R$ and $x \in (2)$, $\tfrac{xy}{2} = y \cdot \tfrac{x}{2} \in R$ since $\tfrac{x}{2} \in R$, which implies $xy \in (2)$ (i.e. $(2)$ is an ideal). Now by Part b, we showed that $R$ is a local ring and $M$ is its maximal ideal.
    \end{proof}
\end{answer}
