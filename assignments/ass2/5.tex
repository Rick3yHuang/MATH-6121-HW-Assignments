\section{Question 5}

\subsection{Part a}

\begin{question}
    Show that $S_n$ is generated by $(12)$ and $(123 \ldots n)$ for all $n \geq 2$.
\end{question}

\begin{answer}
    \begin{proof}
        Let $\sigma = (123\cdots n)$ and $\tau = (12)$. We will show that $(ab)$ can be generated using $\sigma$ and $\tau$ for all $a,b \in \{1,2,3,\cdots,n\}$ using induction on $c = b-a$.

        First, we have $\sigma (12) \sigma^{-1} = (\sigma(1)\sigma(2)) = (23)$. Similarly, we have $(34) = \sigma (23) \sigma^{-1} = \sigma^2(12)\sigma^{-2}$ and $(i(i+1)) = \sigma^{i}(12)\sigma^{-i}$ for $i = 1,2, \cdots n-1$. Now, the base case is proved.
        
        Suppose $(ab)$ for all $a,b \in \{1,2,3,\cdots,n\}$ such that $c = x$ can be generated by $\sigma$ and $\tau$. We claim that $(de)$ for all $d,e \in \{1,2,3,\cdots,n\}$ such that $c = x+1$ can also be generated by $\sigma$ and $\tau$.
        
        Since $((a-1)b) = ((a-1)a)((a-1)b)((a-1)a)^{-1}$ and $((a-1)a) \in \langle \sigma, \tau \rangle$ as proved in the base case, $((a-1)b) \in \langle \sigma, \tau \rangle$ for all $a,b \in \{1,2,3,\cdots,n\}$ such that $c = x$. Also, Since $((a(b+1)) = (b(b+1))((a-1)b)(b(b+1))^{-1}$ and $(b(b+1)) \in \langle \sigma, \tau \rangle$ as proved in the base case, $(a(b+1)) \in \langle \sigma, \tau \rangle$ for all $a,b \in \{1,2,3,\cdots,n\}$ such that $c = x$. Thus, $(de)$ for all $d,e \in \{1,2,3,\cdots,n\}$ such that $c = x+1$ can be generated by $\sigma$ and $\tau$.
        
        Hence, we proved that all transpositions in $S_n$ can be generated by $\sigma$ and $\tau$. Then, because any element in $S_4$ can be expressed using only the transpositions, we know that any element in $S_4$ is in $\langle \sigma, \tau \rangle$.
    \end{proof}
\end{answer}

\subsection{Part b}

\begin{question}
    If $p$ is prime, show that $S_p$ is generated by any transposition and any $p$-cycle.
\end{question}

\begin{answer}
    \begin{proof}
        
    \end{proof}
\end{answer}

\subsection{Part c}

\begin{question}
    Does the conclusion of (b) remain true if we do not assume that $p$ is prime?
\end{question}

\begin{answer}
    No. For example in $S_4 \neq \langle (14),(1423) \rangle$
\end{answer}