\section{Question 7}

\begin{question}
    Let $I, J$ be ideals in a commutative ring $R$ with identity.
\end{question}

\subsection{Part a}

\begin{question}
    Prove that
    $$
    I J=\left\{x_1 y_1+\cdots+x_k y_k: x_1, \ldots, x_k \in I, y_1, \ldots, y_k \in J\right\}
    $$
    is an ideal contained in $I \cap J$. Give an example where $I J \neq I \cap J$.
\end{question}

\begin{answer}
    \begin{proof}
        Let $z \in R$. Now,
        \begin{equation}
            z(x_1 y_1+\cdots+x_k y_k) = zx_1y_1 + \cdots zx_ky_k = (zx_1)y_1 + \cdots (zx_k)y_k \in IJ,
        \end{equation}
        because $I$ is an ideal of $R$, then $zx_i \in I\; \forall i$, and since $R$ is commutative, then
        \begin{equation}
            (x_1 y_1+\cdots+x_k y_k)z \in IJ,
        \end{equation}
        Therefore, $IJ$ is an ideal of $R$. Also, since $I$ and $J$ are the ideals of $R$, then each $x_iy_i \in I\cap J$. Hence, $IJ \subseteq I \cap J$.
    \end{proof}
    Example: $R = \mathbb{Z}$, $I = 2\mathbb{Z}$ and $J = 4\mathbb{Z}$.
\end{answer}

\subsection{Part b}

\begin{question}
    If $I+J=R$, prove that $I J=I \cap J$
\end{question}

\begin{answer}
    \begin{proof}
        Because $R = I+J$ we have:
        $$
        \begin{aligned}
            I &\cap J \subseteq(I \cap J) R\\
            = &(I \cap J)(I+J)\\
            =&I(I \cap J)+J(I \cap J) \text{ (distributive property)}\\
            \subseteq &I J+I J=I J
        \end{aligned}
        $$
    \end{proof}
\end{answer}