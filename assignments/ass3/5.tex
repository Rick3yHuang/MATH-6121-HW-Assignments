\section{Question 5}

\subsection{Part a}

\begin{question}
    How many Sylow 2-subgroups does $S_5$ have?
\end{question}

\begin{answer}
    By Sylow Theorem \#3, know that $n_2 \mid \lvert S_5 \rvert = 120$ and $n_2 \equiv 1 \text{ (mod $3$)}$.
    
    Then, by $n_2 \mid 120$, we know that $n_2 \in \{1, 2, 3, 4, 5, 6, 8, 10, 12, 15, 20, 24, 30, 40, 60, 120\}$. Then considering the restriction $n_2 \equiv 1 \text{ (mod $3$)}$, we could narrow it down to $n_2 \in \{1, 3, 5, 15\}$. If $P \in Syl_2(S_5)$, then $\lvert P \rvert = 8$. Because there are no order $8$ group in $S_5$, we could possibly try to find subgroups of $S_5 \; \cong D_8$. This means $P = \langle \sigma,\tau \mid \sigma^4 = \tau^2 = 1, \sigma\tau = \tau \sigma^{-1} \rangle$. That is we want to find $\sigma$ and $\tau$ such that $\tau \sigma \tau^{-1} = \sigma^{-1}$. $\sigma$ should comes from the $4$-cycles in $S_5$. If we pick $\sigma = (1234)$, then $\sigma = (1432)$, and $\tau$ must be $(1,3)$ then. Hence $\langle (1234),(13) \rangle \leqslant S_5$ and $\lvert \langle (1234),(13)\rangle \rvert = 8$. Similarly, we could find following Sylow 2-subgroups of $S_5$
    \begin{equation}
        \langle (1243),(23) \rangle, \langle (1235),(25) \rangle, \langle (1245),(25) \rangle, \langle (1345),(35) \rangle, \langle (1354),(34) \rangle, \cdots
    \end{equation}
    Hence, $n_2 \geq 6$. Thus, $n_2$ must be $15$.
\end{answer}

\subsection{Part b}

\begin{question}
    Exhibit two distinct Sylow 2-subgroups of $S_5$ and an element that conjugates one to the other.
\end{question}

\begin{answer}
    Let 
    \begin{equation}
        \begin{aligned}
            H_1 &= \langle (1234),(13) \rangle = \{(1),(13),(24),(13)(24),(12)(34),(14)(23),(1234),(1432)\}\\
            H_2 &= \langle (1243),(23) \rangle = \{(1),(23),(14),(14)(23),(12)(34),(13)(24),(1342),(1243)\}
        \end{aligned}
    \end{equation}
    These are two Sylow $2$-subgroups of $s_5$, and one can check that $(12)H_1(12)^{-1} = H_2$.
\end{answer}