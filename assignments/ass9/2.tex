\section{Question 2}

\begin{question}
    Let $R$ be an integral domain and let $M$ be an $R$-module. We define the rank of $M$ to be the maximum number of $R$-linearly independent elements of $M$. We say that $M$ is a torsion $R$-module if for every $x \in M$, there exists a non-zero $r \in R$ such that $r \cdot x=0$.
\end{question}

\subsection{Part a}

\begin{question}
    Suppose that $M$ has rank $n$ and that $x_1, \ldots, x_n$ is any maximal set of linearly independent elements of $M$. Let $N$ be the submodule generated by $x_1, \ldots, x_n$. Prove that $N$ is free of rank $n$ and that the quotient $M / N$ is a torsion $R$-module.
\end{question}

\begin{answer}
    \begin{proof}
        First, since $\mathcal{B} = \{x_1,\cdots,x_n\}$ is a finite linearly independent set that generates $N$. Then, $\mathcal{B}$ is a basis of $N$ and therefore $N$ is a free $R$-module. Also, since $M$ has rank $n$ and $N \subseteq M$ as a submodule, $N$ also has rank $n$. Let $m + N \in M/N$, then $\{m, x_1,\cdots,x_n\}$ will be a linearly dependent set, because $M$ has rank $n$. This means that there exists $r_0,r_1,\cdots,r_n$ which are not all zero such that:
        \begin{equation}
            r_0m + r_1x_1 + r_2x_2 + \cdots + r_nx_n = 0
        \end{equation}
        Therefore, $r_0m = -r_1x_2 - r_2x_2 - \cdots - r_nx_n \in N$. Thus, $r_0(m+N) = r_0m + N = N$. Hence, $M/N$ is a torsion $R$-module.
    \end{proof}
\end{answer}

\subsection{Part b}

\begin{question}
    Prove conversely that if $M$ contains a submodule $N$ which is free of rank $n$ such that the quotient $M / N$ is a torsion $R$-module, then $M$ has rank $n$.
\end{question}

\begin{answer}
    \begin{proof}
        Since $M/N$ is a torsion $R$-module, then if we let $x_1+N,x_2+N,\cdots,x_{n+1}+N \in M/N$, i.e. $x_1,\cdots,x_n,x_{n+1}\in M$, we will know that there exists $y_1,\cdots,y_n,y_{n+1} \in R$ such that $y_1x_1,\cdots,y_nx_n,y_{n+1}x_{n+1} \in N$. Because $N$ is a free module of rank $n$, then $\{y_1x_1,\cdots,y_nx_n,y_{n+1}x_{n+1}\}$ is linearly dependent. Therefore, by definition we could find $r_1,\cdots,r_n,r_{n+1} \in R$ such that:
        \begin{equation}
            r_1y_1x_1 + \cdots + r_ny_nx_n + r_{n+1}y_{n+1}x_{n+1} = 0
        \end{equation}
        Then since $r_1y_1,\cdots r_ny_n,r_{n+1}y_{n+1} \in R$, then $x_1,\cdots,x_n,x_{n+1}$ will not be linearly independent. This means we couldn't find a linearly independent set in $M$ of size bigger than $n$. However, there will be linearly independent set of size $n$ in $N$, which is also in $M$. Then, $M$ has rank $n$.
    \end{proof}
\end{answer}